Blockchains are \emph{monolithic} solutions for the three essentials: \emph{ordering}, \emph{execution} and \emph{finality}.
\sgd{More well-known name for finality is \emph{consensus}, but the meaning of consensus goes blurry as it goes popular.}
This fact itself is not a bad thing, even a good thing: these are all critical for building usable and trustable applications, while an all-in-one is always superior modular (at least in the sense of audience perception).

It is bad for the other fact that, in blockchains, these three functionalities are in \emph{lockstep}.
It orders one, it executes one, and it finalizes one.
And only after what it can order the following ones.
Yes there is batching, but the utilization of batching 1. exactly implies the demand of avoiding lockstep 2. is not a scalable solution and comes with inherent latency penalty.
\sys aims \emph{real time} usages.
So we don't consider the cases that can tolerate long latencies and can be simply solved by batching.

There are always active research and solutions on relieving execution from this lockstep \ie \emph{offchain computation}.
We believe that those solutions are generalized/generalizable so that as long as blockchains can make use of them \sys can do so as well.
This working document mostly focus on the ordering part.

\paragraph{Theoretical limit on improving finality.}
Finality inherently comes with \emph{shared mutable states}, \ie the \emph{decisions} that reach \emph{consensus}.
The mutations on the states must be propagated to the \emph{whole} network \emph{sequentially}, otherwise the states may go diverged.
So we are/will be eventually trapped by the propagation speed on the finality performance.

The alternative workaround is to build \emph{elite networks}.
Under Ethereum context it is the network of stakers (or even just a subset of all stakers that (are allowed to) participant a specific round of proposal).
By shrinking the scope of propagation to just the elites, finality gains more efficiency.
\sgd{Ethereum probably claims the sole purpose of PoS is to be more power efficient than PoW, but I believe this reasoning on the performance of finality must also come into the play.}

This solution implies \emph{homogeneity}.
Only in a homogeneous system we can reach consensus \emph{a priori} on who are elites.
Again under Ethereum context all stakers must stake the same kind of token \ie Ether.
\sys aims \emph{heterogeneous} scenarios.
We believe that in \sys, the mechanism of finality should be everyone \emph{equally}, without any \emph{qualification}, because it's impossible to define the qualification policy reasonably in a heterogeneous system \ie everyone is qualified for different tasks and commitments.
The finality must be propagated to everyone in the same way.

\paragraph{The globally-unique proposer is subjective, and that results in bad quality ordering service.}
\sgd{The argument here needs further streamlining, probably refers the presence of MEV as an evidence, and draw conclusion that ordering must be performed in an intersubjective way.}

\sgd{Eigenlayer's intersubjectivity whitepaper \ie EIGEN token can be a supporting evidence of the necessity of introducing intersubjectivity in ordering mechanism.
It may be worth to consider their ``forkable'' solution as an alternation to \sys and discuss their drawbacks, but hopefully their drawbacks are obvious to many.}

\paragraph{The vision.}
\sys proposes a modular approach for achieving all three of ordering, execution and finality.
Especially, while finality inherently provides ordering \ie based on the order of finalization, we deliberately \emph{deprecate} that order and roll out our own ordering mechanism that is \emph{in real time}, \emph{compatible with heterogeneity} and \emph{intersubjective}.
\sys will eventually be general purpose, but we start small, package it as solutions for certain vertical cases.
\sgd{Somehow we first propose a Bitcoin that later extend it into (or reveal that it is actually) Ethereum.}
This working document focus on the case of financial markets.
We should definitely look at some other cases simultaneously \eg machine learning stuff.